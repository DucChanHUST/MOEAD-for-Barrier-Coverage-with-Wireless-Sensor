\documentclass[preprint,12pt]{elsarticle}

\usepackage{stfloats}
\usepackage{hyperref}
\usepackage[c3]{optidef}
\usepackage{amsmath,amssymb,amsfonts,commath,amsthm}
\usepackage{enumitem}
\usepackage{subcaption}
\usepackage{algorithmic}
\usepackage[ruled, lined, linesnumbered, commentsnumbered, longend]{algorithm2e}
\usepackage{graphicx}
\usepackage{textcomp}
\usepackage{xcolor}
% Draw table from csv file, allow multi-columns a
\usepackage{tabularx}
\usepackage{tabulary}
\usepackage{longtable}
\usepackage{verbatim}
\usepackage{setspace}
\usepackage{csvsimple,array,filecontents,booktabs}

\usepackage[acronym]{glossaries}
\newtheorem{theorem}{Theorem}[section]
\newtheorem{lemma}{Lemma}[section]
\newtheorem{definition}{Definition}[section]
\newtheorem{fact}{Fact}[section]
\newtheorem{pro}{Problem}
\loadglsentries{Glossaries}

\journal{Journal of Network and Computer Applications}

\begin{document}

\begin{frontmatter}

    \title{ Problem Specific MOEA/D for Barrier Coverage with Wireless Sensors}

    \affiliation[1]{organization={Hanoi University of Industry},
        %addressline={298 Cau Dien Street, Minh Khai, Bac Tu Liem District},
        city={Hanoi},
        postcode={100000},
        country={Vietnam}}

    \affiliation[2]{organization={School of Information Technology and Communication, Hanoi University of Science and Technology},
        %addressline={1 Dai Co Viet Street, Bach Khoa, Hai Ba Trung District},
        city={Hanoi},
        postcode={100000},
        country={Vietnam}}


    \begin{abstract}

    \end{abstract}


    \begin{keyword}
        Barrier coverage\sep \textit{k}-strong barrier coverage\sep Wireless directional sensor networks\sep Heterogeneous wireless multimedia wireless sensor networks
    \end{keyword}

\end{frontmatter}

\label{introduction_sec}
\section{Introduction}
In barrier coverage, nodes are deployed in such a manner that it will form a barrier
in a specific path and transmit the information if they sense possible activities made by intruder to cross the barrier. Barrier coverage is applicable to make boundaries of
the critical assets or infrastructure, such as country borders, coastal lines, boundaries
of battlefields, and many more

If a WSN has single connectivity among the sensor nodes then it is called simple connectivity or 1-connectivity. Failure of any single node in such scenario may result in the communication failure in network. Whereas in k-connectivity, in spite of failure of a node, network remains connected by (k-1) number of sensor nodes.
\section{Related works}\label{relatedwork_sec}


\section{Preliminaries and problem formulation}\label{problem}
\subsection{Preliminaries}
\subsubsection{Sensing model}
From literature, two facts about the sensing device can be observed. First one states
that as the distance increases the sensing ability decreases. Second, sensing ability
gets improved as the noise reduces. Mainly, two types of sensing models can be
found in literature based on the detection probability, (i) binary disk-sensing model
and (ii) probabilistic sensing model. This paper focuses on the second one. Probabilistic sensing model is given by following equation \ref{eq1}

\begin{equation}
    \label{eq1}
    \mathcal{C}(O, s_i) = \begin{cases}
        1                          & {\rm{if } } \quad  d(O,s_i) \le R_S +R_e             \\
        -e^{{\alpha\beta^\lambda}} & {\rm{if } } \quad  R_S-R_E \le d(O,s_i) \le R_S+ R_e \\
        0                          & {\rm{if } } \quad  d(O, s_i) \ge R_S+R_e             \\
    \end{cases}
\end{equation}

where certainty detection of the sensor nodes is defined by $R_e$ ($R_e$ < $R_s$), $\alpha$, $\beta$, and $\lambda$
are the detection probability at distance less than or equal to $R_S$ and
$\beta =
    d(O, s_i) - (R_S - R_e)$.
Equation \ref{eq1} depicts that, a point is said to be covered if it lies within a distance $(R_S - R_e)$ from a sensor node, coverage diminishes exponentially as the distance between points and the sensor nodes increases if points are lying within an interval $(R_S - R_e, R_s + R_e)$. The points lying beyond distance $(R_S + R_e)$ are said to be uncovered. In a network, it may be possible that a point might be covered by multiple (k) sensor nodes then a point is known as k-covered.
\subsubsection{Communication model}
The simplest communication model which is mostly found in the literature is binary
disk model. Like the sensing range (RS), communication range (RC) is defined as the
range up to which it can communicate with other nodes. Based on the transmission
power level, different sensor nodes have different communication range. Two sensor
nodes can communicate with each other if they are within the RC

\subsection{Problem formulation}
In this paper, we address the problem of sensor deployment for continuous monitoring of a linear segment on the x-axis with length L. We consider N static sensors that can be either activated or deactivated. These sensors are randomly distributed over the monitored area using helicopters or unmanned aerial vehicles (UAVs). Each sensor $s_i$ has a unique position $x_i$ and a specific sensing radius $r_i$. With this radius, the coverage range of sensor $s_i$ can be represented as the interval $C(s_i, r_i)$ of length $2r_i$ centered at $x_i$. Additionally, each sensor is associated with a cost $c(s_i, r_i)$, this cost represents the energy consumption during opearion. The objective is to choose a coverage radius $r_i$ for each sensor $s_i$ such that
\begin{align*}
    L & \subseteq \bigcup_{s \in S} C(s, r)
\end{align*}
and the total cost of the deployment $\sum_{s\in S}c(s, r)$ is minimized. This optimizationproblem is referred to as min-cost-linear-coverage (MCLC).

Decision variable 1: using status of sensor $s_i$
\begin{equation}
    {\mu_i} =
    \begin{cases}
        1 & {\rm{if } } \quad \text{sensor} \, s_i \text{ is active} \\
        0 & {\rm{otherwise}}
    \end{cases}
\end{equation}

Decision variable 2: Sensing range of sensor $s_i$
\begin{equation}
    0 \leq {r_i} \leq {r_{max}}
\end{equation}


Objective 1 (reliability): Minimizing the number of active sensors:
\begin{equation}
    f_1 = \sum_{s_i \in S} \mu_i
\end{equation}

Objective 2 (Power): Minimizing the total power consumption:
\begin{equation}
    f_2 = \sum_{s_i \in S} p \cdot {r_i}^\alpha
\end{equation}


To ensure that the constructed barrier can effectively detect intruders moving across the monitored area, the following constraints must be satisfied:

Constraint 1: Every point $p$ of the barrier is covered by at least one active
used sensor $s_i$.

\section{Proposed algorithm}\label{algorithms_sec}
\subsection{MEOA/D}
\subsection{NSGA-II}

\section{Experimental results}
\subsection{Parameter settings}



\subsection{Experimental results}
\section{Conclusion}\label{conclusion_sec}

\appendix

\bibliographystyle{elsarticle-num}
\bibliography{references}
\end{document}
\endinput
